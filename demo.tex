\documentclass{article}

\author{Shubham Sharma}
\title{My first {\LaTeX} Document}

\begin{document}

\maketitle
\section{Introduction}

This is the first time that I'm on Share\LaTeX. This is a paragraph intentionally kept long in order to see the line widths and paragraph formatting.

This is a newline.
This ain't a newline. asfasjdhfkj askjdfhkj asdkjf as dfh jasdh ffkl asdkhfkjlash df awiueh fh fucniuoew cj nh unh cjen.

This is another paragraph and is intentionally kept long again in order to see the line formatting and paragraph formatting.
\subsection{Intro summary}
First line in a section always starts without indentation. xcvjasklfj aklsjd fjaskld fowe fniasjd fwejfiow eflj weiof wee.

Any subsequent paragraphs start with an indent of 4 spaces. lkajsdf asdklf asjfw eifj w09fj ioe fiowj efj weijf iojdf ajsd f. Just like this one.

And this one too.

\section{Formatting}
This is normal text.

\textbf {This text is bold text!}

\emph {This text is emphasized.}

If no curly braces are used, only the first letter after bold or emph command will get the effect.

\emph Like this. \textbf And this.
\textit{textit works the same way as emph.}

The only difference is what purpose the two are being used for. Italics should be used where the text always has to be italicized, like Book Titles, Author names, etc.

On the other hand, emph should be used with text that isn't always supposed to be that way. For instance, a style-guide might require some text that is emphasized to rather be bold or underlined (not a book title or an author, obviously).
This way we can differentiate among the two and make changes to the document accordingly without much pain.

\underline{This text is underlined.}

\subsection{Quotes}
Opening quote is the \textbf{grave accent} and closing quote is the \textbf{apostrophe}.

"Improper double quotation marks"

``proper double quotation marks''

'improper single quotation marks'

`proper single quotation marks'

\end{document}
