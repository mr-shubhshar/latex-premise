\documentclass{article}

\author{Shubham Sharma}

\title{Intro to Lists, Refs and Labels in {\LaTeX}}

\begin{document}

\maketitle

\section{Introduction}
In section \ref{lists}, there are 2 lists: 1 ordered and 1 unordered each containing 1 unordered and 1 ordered sublist, respectively.

The section \ref{lnr} is about Labels and References.

\section{Lists\label{lists}}

Below is an ordered list:

\begin{enumerate}
    \item{Bananas}
    \item{Papaya}
    \item{Bread}
    \begin{itemize}
        \item{White bread}
        \item{Brown bread}
    \end{itemize}
    \item{Chawal}
\end{enumerate}

Below is an unordered list:

\begin{itemize}
    \item Tee
    \item Socks
    \begin{enumerate}
        \item size L
        \item size M
    \end{enumerate}
    \item Gloves
\end{itemize}

\section{Labels and Refs\label{lnr}}

While editing a document, say you have to change the order of sections or lists or anything that might affect their indexes, if you hard-code the indexes, that might be troublesome since you might have to re-type all those indexes again.

The solution to this is Labels and References. {\LaTeX} will automatically perform these tasks for you if you've labelled every section and/or list items.

\end{document}
